\section{How to generate a new base system?}

\subsection{install necessary system files}
\begin{verbatim}
debootstrap --arch=i386 woody .
\end{verbatim}
This may run in to an error, but it doesn't matter. Simply follow this guide ;) .

\subsection{editing files}
\subsubsection{fstab}
Copy your existing /etc/fstab to the new /etc/ directory. Adjust the line for your boot device, e.g. your new boot device is hda3 so you have to change \textit{/dev/hda2       /               ext3   defaults           0       0} to \textit{/dev/hda1       /               ext3   defaults           0       0} .
Your new fstab may look like this:
\begin{verbatim}
# /etc/fstab file system info, created by the m23 project
#  <file system> <mount point>  <type> <options>     <dump>  <pass>
proc             /proc          proc   defaults      0       0
/dev/hda1        none           swap    sw           0       0
/dev/hda3       /               ext3   defaults           0       0
/dev/cdrom      /mnt/cdrom      auto    ro,noauto,user,exec     0 0
/dev/fd0        /mnt/floppy     auto    user,noauto             0 0
\end{verbatim} 

\subsubsection{lilo.conf}
You can use your current lilo.conf and edit some values like \textit{root}. You may choose a different kernel than in your current system so you have to adjust the \textit{image} parameter too. But first it is essetial to have a lilo.conf with the correct \textit{root} value, \textit{image} doesn't matter at the moment.
\begin{verbatim}
boot=/dev/hda
map=/boot/map
install=/boot/boot-menu.b
compact
prompt
vga=normal
delay=10
timeout=20

default=m23Server3

image=/boot/vmlinuz-2.4.20-1-386-sec
        root=/dev/hda3
        label=m23Server3
        append="hdb=scsi-ide"
        initrd=/initrd.img
\end{verbatim}

\subsubsection{sources.list}
Your sources.list should look like this:
\begin{verbatim}
 deb http://security.debian.org/ woody/updates main contrib non-free
 deb http://ftp.de.debian.org/debian/ stable main non-free contrib
 deb http://ftp.szczepanek.de/ftp/trusteddebian/ stable main contrib
\end{verbatim}
You can add more sources if preferred.

\subsubsection{/etc/network}
Simply copy your old /etc/network directory to the new /etc.

\subsection{step in to your new system}
You can enter your new system (without reboot) with:
\begin{verbatim}
chroot .
\end{verbatim}

\subsection{source update and cleanup}
I think we don't need mailx :) .
\begin{verbatim}
apt-get update
apt-get remove mailx
\end{verbatim}

\subsection{installing a kernel}
You can list all available kernels with the following command:
\begin{verbatim}
apt-cache search kernel | grep kernel-image
\end{verbatim}
Pick one out adjust the \textit{image} value in your new \textit{lilo.conf}. To install your new kernel type the following:
\begin{verbatim}
apt-get install myChosenKernel-Image
\end{verbatim}

\subsubsection{in case of an error}
If you get an error with the \textit{mkinitrd} command try it manualy. Adjust \textit{/boot/initrd.img-2.4.20-1-386-sec} to the filename you entered in the \textit{lilo.conf}. You have to adjust the \textit{/lib/modules/...} and the root device after \textit{-r} also.
\begin{verbatim}
mkinitrd -r /dev/hda3 -o /boot/initrd.img-2.4.20-1-386-sec /lib/modules/2.4.20-1-386-sec/
\end{verbatim}

\subsection{system update and installation of additional packages}
\begin{verbatim}
apt-get dist-upgrade -u
apt-get install ssh
\end{verbatim}

\subsection{booting your new system}
To see if your new system is ready to boot, copy your new kernel from the new boot/ directory to /boot. This kernel will be called like \textit{vmlinuz-2.4.20-1-386-sec}.
Edit your \textbf{current} /etc/lilo.conf and add a new section like this:
\begin{verbatim}
image=/boot/vmlinuz-2.4.20-1-386-sec
        root=/dev/hda3
        label=m23Server3
        initrd=/initrd.img
\end{verbatim}
vmlinuz-2.4.20-1-386-sec should be replaced with the name of your kernel and the section above has to be the same as in your new \textit{lilo.conf}.