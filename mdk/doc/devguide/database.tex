\section{overview}
m23 stores all information about the clients, their status, pending jobs and hardware infos in a database. This chapter will show you how the m23 database is organized. The m23 database is divided in several tables:
\begin{itemize}
\item clientjobs: stores waiting and done jobs for each client. If you install a package to a client the information about the install job is stored in this table with the status waiting. When the package is installed the install job will be marked as "done".
\item clientlogs: Here are stored error and success of the installation procedure. You can see the output of whole installation procedure.
\item clientpackages: here are stored all packages installed on the clients with status, version and action.
\item clientpreferences: to make adding a new client more easy you can save preferences for new clients. These preferences are stored here.
\item clients: information about all clients, ip, ram size, cpu, etc.
\item ftpusers: will later be used for access to the ftp server on the m23 server.
\item groups: clients can be organized in groups to manage them more easy.
\item packages: will be used for storing available packages that can be installed on the clients.
\item plugins: information about installed plugins.
\item recommendpackages: here you can store package selections. With a package selection you can install a bundle of software with one click on a client.
\item remotevar: table to store the remove vars.
\end{itemize} 

\section{the tables}
\subsection{clientjobs}
If you install or deinstall software on the client, this (de)install job will be saved in the clientjobs table. The new job is added with the status "waiting" and after succesful finish it will get the status "done". Every job can have serveral parameters, e.g. if you want to format a partition, the format job should know which partition to format. This infomation is stored as the parameter. And of course each job should know the name of the client it is for. To get the jobs in the correct execution order, there are two values: id and priority. Priority has to do with the type of the job, a special job like the "hardware scan" gets the priority 0 and should be executed before all other jobs. With the priority the order of execution is set, lower priorities are executed earlier. The second value "id" is the order of job creation. Jobs of the same priority will get executed in the order of creation. Remember: priority is mightier than the id value. If the priority is lower the job will be executed earlier while the id might be higher.
\begin{itemize}
\item id: the id of the job
\item client: name of the client, the job is for.
\item package: name of the package
\item priority: the priority of the package
\item status: status of the job: waiting, done
\item params: the parameters for the job
\end{itemize}

\subsection{clientlogs}
clientslogs saves the information about the installation of clients. The input is generated by the log2db tool and PHP scripts.
\begin{itemize}
\item client: name of the client, the log information is saved for.
\item logtime: time the log event was.
\item status: contains the logged information.
\end{itemize}

\subsection{clientpackages}
\begin{itemize}
\item clientname: name of the client, the package is installed on
\item package: name of the package
\item version: version of the package
\item status: is status status and can be "install ok installed", "deinstall ok config-files" and every other status, debian packages can have.
\item action: action tells what should be done with the package. Possible actions can be: none, remove and reinstall.
\end{itemize}

\subsection{clientpreferences}
A preference can store multiple variables with its values, all of these variables are stored with the same preference name.
\begin{itemize}
\item name: name of the preference
\item var: name of preference variable
\item value: value for the variable
\end{itemize}

\subsection{clients}
The clients table stores information about hardware, the network settings, the username, email, etc. .
\begin{itemize}
\item client: the name of the client
\item office: here you can leave information about the place where the client stands.
\item name: name of the user
\item familyname: familyname of the user
\item eMail: eMail address for the user
\item mac: mac address of the network card
\item ip: ip address
\item netmask: netmask for the ip
\item gateway: gatewas address
\item dns1: ip address of the first domain name server 
\item dns2: ip address of the second domain name server 
\item groupname: name of the group the client is in
\item firstpw: the password for the first login
\item rootPassword: root password
\item memory: size of the installed memory in MB
\item hd: size of harddisk in MB
\item partitions: data about the partitions
\item cpu: type of cpu
\item MHz: speed of the cpu
\item netcards: product names of the installed network cards
\item graficcard: information about the grafic card
\item soundcard: name of the sound card
\item isa: information about ISA components
\item dmi: DMI information
\item dhcpBootimage: name of the currently used bootimage
\item installdate: date the clients first was set up
\item lastmodify: date the client was last modified
\item status: actual status of the client. 0: client has not finished the hardware detection sequence. 1: client has finished hardware detection and waits for partition/format job 2: the client is partitioned and formated and has installed the base system.
\end{itemize} 

\subsection{ftpusers}
This table can store information about ftpusers with their permissions.
\begin{itemize}
\item Password: password for th user
\item Uid: user ID
\item Gid: group ID
\item Dir: directory for the user
\item QuotaFiles: quota for the amount of files.
\item QuotaSize: quota for the whole size of all files
\item ULRatio / DLRatio: ratio for upload to download, you have to set both values
\item ULBandwidth: max speed for upload
\item DLBandwidth: max speed for download
\item User: name of the FTP user
\end{itemize}

\subsection{groups}
With groups you will be able to organize your clients more efficient. You can add a client to agroup and install software on all clients of a group without selecting each of the clients. It just saves some clicks ;)
\begin{itemize}
\item groupname: name for the group
\end{itemize}

\subsection{plugins}
With plugins you can enrich you m23 admin console with additional functionalities. Plugins are a bundle of PHP/Bash and other files showing one or more dialogs in the m23 admin, that are designed for a special purpose. E.g. you can write a backup plugin that lets you backup all m23 clients. For more information abou plugins see the "How to develop plugins for m23?" chapter.
\begin{itemize}
\item name: name of the plugin
\item author: who had done it?
\item version: version number
\item updateurl: where to get the update file. An update file contains information about the new plugin and the plugin data itself.
\item clientRequires: packages that have to be installed on the client before you can use the plugin. E.g. if you install a backup plugin there shoulb be installed the backup software on the client.
\item deinstall: here is stored the uninstall script, this is normaly a Bash script.
\item files: the file names included in the plugin, this is saved for clean uninstall.
\item installdate: when was it installed?
\end{itemize}

\subsection{recommendpackages}
In the recommendpackages table are stored package selections for reuse at a later moment. E.g. you may save a selection containing OpenOffice, Mozilla and Gimp for office usage. Now you can install these three packages with the selection and don't have to install each of them.
\begin{itemize}
\item name: name of your selection (e.g. office)
\item package: the name of the package in the selection (e.g. openoffice.org)
\item version: may be used later if we have to select between different versions of a package.
\item priority: the priority of a package selects when to install the package among other packages in the whole installation process. Packages with lower numbers are installed earlier.
\item params: special parameters for the package.
\end{itemize}

\subsection{remotevar}
With remote variables you can store values server side. The variables are stored for a special ip.
\begin{itemize}
\item ip: the ip address the variable is stored for.
\item var: name of the variable
\item value: the value for the variable
\item addtime: the time the value was changed / added
\end{itemize}