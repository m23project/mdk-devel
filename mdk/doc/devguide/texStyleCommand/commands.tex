\newcommand{\bildrechts}[1]{ %#1: Bild, das nach rechts flie�en soll
\begin{wrapfigure}{r}{0.2\textwidth}
		\includegraphics[width=0.19\textwidth]{#1}
\end{wrapfigure}
}

\newcommand{\code}[1]{ %#1 kurzer Quelltext
	\ausgabe{#1}
}



\newcommand{\beschreibung}[1]{ %#1 Beschreibungstext
	\\ {\bf Beschreibung:} \textit{#1} \\
}



\newcommand{\hinweis}[1]{ %#1 Hinweistext
	\blaukasten{Hinweis}{#1}
}



\newcommand{\portierung}[1]{ %#1 Portierungstext
	\\ \blaukasten{Portierung}{#1} \\
}



\newcommand{\fragen}[1]{ %#1 Fragentext
	\\ \gelbkasten{Fragen}{#1} \\
}



\newcommand{\problem}[1]{ %#1 Problemtext
	\\ \rotkasten{Problematisch}{#1} \\
}



\newcommand{\rotkasten}[2]{ %#1:�berschrift #2:Text
	\smallskip
	\begin{tabularx}{17cm}{X}\hline
	\rowcolor[cmyk]{0.0,0.92,1.0,0.34}
	\bf \sffamily \normalsize \textcolor{psrotfnt}{#1} \\ \hline
	\rowcolor[cmyk]{0.0,0.02,0.10,0.0}
	\normalsize \sffamily #2 \\ \hline
	\end{tabularx}
	\smallskip
}

\newcommand{\blaukasten}[2]{ %#1:�berschrift #2:Text
	\smallskip
	\begin{tabularx}{17cm}{X}\hline
	\rowcolor[cmyk]{0.44,0.16,0.0,0.10}
	\bf \sffamily \normalsize \textcolor{psrotfnt}{#1} \\ \hline
	\rowcolor[cmyk]{0.0,0.02,0.10,0.0}
	\normalsize \sffamily #2 \\ \hline
	\end{tabularx}
	\smallskip
}

\newcommand{\gelbkasten}[3]{ %#1: width #2: Rand #3: Text
	\smallskip
	\begin{tabularx}{17cm}{X}\hline
	\rowcolor[cmyk]{0.0,0.01,0.63,0.18}
	\bf \sffamily \normalsize \textcolor{psrotfnt}{#1} \\ \hline
	\rowcolor[cmyk]{0.0,0.02,0.10,0.0}
	\normalsize \sffamily #2 \\ \hline
	\end{tabularx}
	\smallskip
}

\newcommand{\ausgabe}[1]{ %#1: code
	\ \\
	\medskip
	\begin{tabularx}{17cm}{X}\hline
	\rowcolor[rgb]{0.8,0.8,0.8}
	\bf \sffamily \normalsize \textcolor{psrotfnt}{Ausgabe} \\ \hline
	\rowcolor[rgb]{0.9,0.9,0.9}
	\normalsize \sffamily \verb�#1� \\ \hline
	\end{tabularx}
	\smallskip
}

\newcommand{\entspr}[0]{
$\mathop{\hat{=}}$
}


\newcommand{\paket}[1]{ %#1: code
	"\emph{#1}"
}



%Marks a command (without return key) into the text
\newcommand{\kommandoOR}[1]{ %#1: code
	\fboxsep0.6mm
	\colorbox{bgGray}{\fbox{\textbf{\tt{#1}}}}
}



%Marks a command into the text
\newcommand{\kommando}[1]{ %#1: code
	\fboxsep0.6mm
	\colorbox{bgGray}{\fbox{\textbf{\tt{#1} \keys{\return}}}}
}





%Includes a short DIFF file with syntax highlighting
\newcommand{\includeDIFF}[1]{ %#1: codefile
	\ \\
	\\
	\inputminted[linenos, numbersep=5pt, frame=lines, framesep=2mm, bgcolor=bgGray, fontseries=sffamily, funcnamehighlighting=true, obeytabs=true, fontsize=\footnotesize]{diff}{#1}
	\smallskip
}





%Includes a short BASH file with syntax highlighting
\newcommand{\includeBASH}[1]{ %#1: codefile
	\ \\
	\\
	\inputminted[linenos, numbersep=5pt, frame=lines, framesep=2mm, bgcolor=bgGray, fontseries=sffamily, funcnamehighlighting=true, obeytabs=true, fontsize=\footnotesize]{bash}{#1}
	\smallskip
}





%Includes a short HTML file with syntax highlighting
\newcommand{\includeHTML}[1]{ %#1: codefile
	\ \\
	\\
	\inputminted[linenos, numbersep=5pt, frame=lines, framesep=2mm, bgcolor=bgGray, fontseries=sffamily, funcnamehighlighting=true, obeytabs=true, fontsize=\footnotesize]{html}{#1}
	\smallskip
}





%Includes a short PHP file with syntax highlighting
\newcommand{\includePHP}[1]{ %#1: codefile
	\ \\
	\\
	\inputminted[linenos, numbersep=5pt, frame=lines, framesep=2mm, bgcolor=bgGray, fontseries=sffamily, funcnamehighlighting=true, obeytabs=true, fontsize=\footnotesize]{php}{#1}
	\caption{#1}\\
	\smallskip
}





%Marks a command into the text
\newcommand{\datei}[1]{ %#1: code
	\textbf{\tt{#1}}}


\newcommand{\sieheSeite}[1]{ %#1: Sprungmarke zu einem label
	(siehe Seite \pageref{#1})}


\newcommand{\siehe}[1]{ %#1: Sprungmarke zu einem label
	(siehe \ref{#1})}


\newcommand{\link}[1]{ %#1: Linkadresse
\footnote{\url{#1}}}