\chapter{Erste Schritte mit dem m23-Server}
\section{Verbindung zum m23-Server aufnehmen}
Nachdem Sie Ihren Server erfolgreich installiert und neu gestartet haben, gelangen Sie �ber einen beliebigen Webbrowser auf einem am Netzwerk angeschlossenen PC auf die m23-Administrationsoberfl�che. Geben Sie dazu in Ihrem Webbrowser folgende Ziel-URL ein:\\\\
\underline{http://ServerIP}\\\\

Als ServerIP nehmen Sie bitte die IP Ihres m23-Servers (z.B.: 192.168.1.23). Beim ersten Einloggen haben Sie noch kein eigenes m23-Administrator-Konto, deshalb geben Sie bitte das LOGIN und PASSWORT ein, welches auf der Hintergrundfl�che steht. Sobald Sie eingeloggt sind, sollten Sie sich ein eigenes m23-Administrator-Konto einrichten und das god-Konto l�schen.

\section{Vorgehen beim Hinzuf�gen eines Clients}
Hier wird das allgemeine Vorgehen beim Hinzuf�gen eines neuen Clients beschrieben. F�r die genaue Vorgehensweise sei auf die jeweiligen Kapitel verwiesen.

\begin{itemize}
\item Schlie�en Sie den Client an das Netzwerk an, so da� Client und Server aufeinander zugreifen k�nnen.
\item Erstellen Sie ggf. eine Bootdiskette f�r den Client und legen Sie diese ein. Eine Bootdiskette ben�tigen Sie nur, falls Ihr Client nicht �ber eine Netzwerkkarte mit PXE verf�gt.
\item Starten Sie den Client und notieren Sie sich die MAC-Adresse (z.B. 00:45:23:3A:96:F3).
\item Geben Sie die MAC-Adresse zusammen mit den anderen ben�tigten Daten ein (wie im Kapitel "Clients verwalten" beschrieben) und legen Sie den Client an.
\item Sollte der Client nicht automatisch booten, dann Rebooten Sie ihn und lassen die evtl. erstellte Bootdiskette im Laufwerk.
\item Nun bootet der Client und sendet Hardware- und Partitionierungs-Informationen an dem m23-Server.
\item Im Webinterface k�nnen Sie diese Daten begutachten. Fahren Sie mit dem Kapitel "Clients partitionieren und formatieren" und anschlie�end mit "Installation des Betriebssystems" fort.
\item Zur Installation zus�tzlicher Software schauen Sie in das Kapitel "Pakete installieren/deinstallieren".
\end{itemize}