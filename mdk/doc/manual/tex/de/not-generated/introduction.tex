\chapter{Einleitung}
\section{Was ist m23?}
m23 ist eine freie Softwareverteilung unter der GPL, die Clients mit Debian, Ubuntu und Kubuntu installieren und administrieren kann. Gesteuert wird m23 mit einem Browser. Die Installation eines neuen Clients geschieht in nur drei Schritten, die Integration von bestehenden Clients ist zudem m�glich. Gruppenverwaltung und Masseninstallation vereinfachen die Administration von vielen Rechnern. Das integrierte Client- \& Serverbackup sch�tzt vor Datenverlusten. Mittels Virtualisierung k�nnen auf m23-Client und m23-Server weitere virtuelle m23-Clients angelegt und �ber m23 verwaltet werden. Skripte und Softwarepakete zur Installation auf den Clients k�nnen direkt aus der Oberfl�che erstellt werden.



\section{Die Bestandteile des m23-Servers}
\subsection{Debian GNU/Linux}
\begin{center}
\includegraphics{/mdk/doc/manual/screenshots/debianlogo.png}
\end{center}
m23 basiert vollst�ndig auf Debian GNU/Linux \underline{www.debian.org}, einer freien Linux-Distribution. Sowohl der Server als auch die m23-Clients sind Debian GNU/Linux-Systeme. Wir haben bei der Entwicklung von m23 auf Transparenz aller Vorg�nge und besonders auf lizenzfreie Clients und Server geachtet! Debian bietet derzeit ca. 25.000 Softwarepakete, die aus dem Internet heruntergeladen und auf beliebig vielen Clients installiert werden k�nnen.




\subsection{Auf die Schultern von Giganten stellen}
m23 besteht aus einer Vielzahl von OpenSource-Komponenten, ohne die es einige Jahrzehnte gedauert h�tte, m23 in dieser Form zu entwickeln. Allein durch die Arbeit unz�hliger Freiwilliger ist m23 m�glich geworden. Vor allem sind es folgende Programme, die in m23 zum Einsatz kommen:
\begin{itemize}
\item Apache: wird von m23 als Datei-Server und f�r die Generierung der m23-Administrations-Oberfl�che und Skripte benutzt.
\item PHP: ist die am h�ufigsten in m23 verwendete Programmiersprache: Die Oberfl�che sowie die komplette Skriptgenerierung sind in PHP geschrieben.
\item MySQL: dient der Speicherung der meisten Daten, die von m23 verwaltet werden.
\item Etherboot: kann als alternative Methode zum Starten der Clients �ber das Netzwerk verwendet werden.
\item DHCP, ATFTP: machen das Booten der Clients erst m�glich.
\item BusyBox: wird f�r das Mini-System der Installations-CD sowie das Netzwerk-Bootimage verwendet, um eine Menge Platz zu sparen.
\item und viele mehr...
\end{itemize}


\section{Lizenz}
m23 ist ein OpenSource-Produkt, das unter der GPL (GNU General Public License) steht! Die genauen Lizenzbedingungen der GPL sind im Anhang des Buches enthalten. Das beutet im Groben f�r Sie:
\begin{itemize}
\item Die Nutzung ist kostenlos.
\item Der komplette Quellcode ist verf�gbar.
\item Sie d�rfen den Quellcode ver�ndern.
\item Sie d�rfen m23 kopieren und weitergeben, so oft Sie wollen.
\item \textbf{�nderungen von Ihnen an m23 m�ssen ebenfalls unter der GPL ver�ffentlicht werden!}
\end{itemize}

\section{Download}
Da Sie dieses Handbuch gefunden haben, wissen Sie wahrscheinlich schon, wo man m23 herunterladen kann ;). Aber falls nicht:\\
Das ISO zum Brennen der m23-Server-Installations-CD sowie Dokumentationen und Debian-Pakete (zu einem sp�teren Zeitpunkt) gibt es unter \underline{http://m23.sf.net}.



\section{Support}
\subsection{m23-Community-Support}
Sie erhalten einen kostenlosen m23-Community-Support auf freiwilliger Basis. Das bedeutet aufgrund der vielen Anfragen, da� nicht jedes Anliegen zur Zufriedenheit aller gel�st werden kann. Der Community-Support deckt zudem lediglich m23-Installationshilfen in Standard-Konfigurationen sowie das Korrigieren von m23-Programmfehlern ab.\\\\

Der m23-Community-Support verlangt vom m23-Nutzer zudem gute Kenntnisse von Debian, die Bereitschaft genaue Fehlerdiagnosen zu liefern und gemeinsam mit dem m23-Entwickler das Problem zu l�sen.\\\\

Sollten Sie mehr ben�tigen, so wenden Sie sich bitte an den kommerziellen Support.\\\\

Auf der m23-Community-Seite finden Sie:
\begin{itemize}
\item Anwenderforen (f�r den Community-Support)
\item Dokumentationen
\item ISO-Images zum Erstellen der Installations-CDs
\item Community-Newsletter
\item u.v.m.
\end{itemize}




\subsection{Kommerzieller Support}
Der kommerzielle Support  bietet Ihnen Dienstleistungen rund um m23 und dar�ber hinausgehende Serviceleistungen an:
\begin{itemize}
\item m23-Schulungen f�r Entwickler und Anwender in kleinen Gruppen
\item Entwicklung von individuellen m23-Erweiterungen und m23-Anpassungen
\item Beratung: Unter anderem bei der Konzeptionierung von Clients, Servern und Netzwerken f�r den Einsatz von m23 sowie allgemeine OpenSource-Beratung
\item Erstellung virtueller Maschinen und virtueller Appliances
\item Design und Entwicklung von Webseiten auf Basis des Content-Management-Systems devalcms
\item Services rund um die Webseiten-Capture-Software khtml2png
\item Erstellung und Konzeptionierung von Online-Wahlsystemen basierend auf dem GRDM-System
\item Schulungen zu Linux, OpenSource, Skriptprogrammierung mit der BASH, awk, sed, grep und Co. sowie Schulungen f�r HTML, PHP und MySQL
\end{itemize}
Weitere Informationen finden Sie unter \underline{www.goos-habermann.de}.

	


\section{Hinweis zum Handbuch}
Mit der neuen m23-Version geht auch das Benutzerhandbuch in eine neue Runde. Um es f�r Entwickler leichter zu machen, wird nun ein gro�er Teil des Handbuches aus den Hilfetexten des m23-Webinterfaces generiert. Dies hat zum einen den Vorteil, da� die Dokumentation nicht zweimal erstellt bzw. kopiert werden mu� (Web + Handbuch) und zum anderen ist so ein Handbuch in allen Sprachen m�glich, die vom Webinterface unterst�tzt werden.
