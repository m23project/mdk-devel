\chapter{Premiers pas avec le serveur m23}
\section{Entrer en contact avec le serveur m23}
Apr�s avoir install� votre serveur avec succ�s et l'avoir d�marr� de nouveau, vous atteignez l'interface d'administration de m23 par un navigateur quelconque d'un ordinateur connect� au r�seau. Entrez l'adresse web suivante dans votre navigateur web:\\\\
\underline{http://ServerIP}\\\\

Comme ServerIP employez le num�ro IP de votre serveur, s.v.p. (par exemple 192.168.1.23). � la premiere entr�e au syst�me vous n'aurez pas encore un compte d'administrateur propre, pour cette raison entrez le nom (LOGIN) et le mot de passe (PASSWORD) que vous voyez dans l'arri�re-plan. Aussit�t que vous ayez entr� le syst�me, vous devriez �tablir un compte d'administrateur propre et effacer le compte $\ll$god$\gg$.

\section{Proc�d� pendant l'ajout d'un client}
Ici, le proc�d� g�n�ral pendant l'ajout d'un nouveau client est d�crit. Pour le proc�d� exacte, regardez  les chapitres respectifs. 

\begin{itemize}
\item Connectez le client au r�seau pour que le client et le serveur peuvent acc�der l'un l'autre.
\item Le cas �ch�ant, faites un disque pour d�marrer pour le client et ins�rez-le. Vous en avez seulement besoin, si votre client n'a pas de carte de r�seau avec PXE.
\item D�marrez le client et notez l'adresse MAC (par exemple 00:45:23:3A:96:F3).
\item Entrez l'adresse MAC et les autres donn�es n�cessaires (comme c'est d�crit dans le chapitre $\ll$Administrer des clients$\gg$) et �tablissez le client.
\item Si le client n'amorce pas automatiquement, ramorcez-le et laissez le disque pour d�marrer que vous avez �ventuellement fait dans le lecteur.
\item Maintenant, le client amorce et envoie des information sur le mat�riel et le partitionnement au serveur m23.
\item Dans l'interface web, vous pouvez �valuer ces donn�es. Continuez avec le chapitre $\ll$Formater et partitionner des clients$\gg$ et apr�s avec $\ll$Installation d'un syst�me d'exploitation$\gg$.
\item Concernant l'installation du logiciel suppl�mentaire, regardez le chapitre $\ll$Installer et d�sinstaller des paquets$\gg$.
\end{itemize}