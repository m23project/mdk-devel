\chapter{Introduction}
\section{What is m23?}
m23 is a free software distribution system licensed under the GPL, that installs and administrates clients with Debian, Ubuntu, Kubuntu. m23 is controlled via webbrowser. The installation of a new m23 client is done in only three steps and the integration of existing clients is possible, too. Group functions and mass tools make managing a vast number of clients comfortable. Client backup and server backup are included to avoid data loss. With the integrated virtualisation solution, m23 can create and manage virtual m23 clients, that are run on real m23 servers and m23 clients. Scripts and software packages (for installation on the clients) are created directly from the m23 webinterface. 

\section{The components of the m23 server}
\subsection{Debian GNU/Linux}
\begin{center}
\includegraphics{/mdk/doc/manual/screenshots/debianlogo.png}
\end{center}
m23 is based on Debian GNU/Linux \underline{www.debian.org}, a free Linux distribution. Debian GNU/Linux is used for the m23 clients, too. We have payed attention to transparency of all procedures and license free clients and servers during development! Debian offers about 25.000 software packages that can be downloaded from the internet and installed on any client you want.




\subsection{Place the development on the shoulders of giants}
m23 is composed of a multitude of OpenSorce components. Without these components the development of m23 would have taken decades. The work of countless volunteers made m23 possible. m23 uses the following programs among which are most notably:
\begin{itemize}
\item Apache: is used as file server and for the generation of the web interface and scripts.
\item PHP: a common used script language: m23 uses it to generate the web interface and scripts for installation etc.
\item MySQL: stores most of the data m23 manages.
\item Etherboot: an alternative method to boot clients over the network.
\item DHCP, ATFTP: makes the network booting possible.
\item BusyBox: is used for the mini system of the installation CD and the network bootimages. BusyBox saves a lot of space.
\item many more...
\end{itemize}


\section{License}
m23 is an OpenSource solution that is licensed under the GPL (GNU General Public License)! The full license text of the GPL can be found in the appendix of this book. In simple words, the GPL means:
\begin{itemize}
\item usage is free of charge.
\item the complete sourcecode is available.
\item you may change the sourcecode.
\item you can copy and share m23 as often and with as many people as you like.
\item \textbf{Changes you make to m23 have to be put under the GPL again!}
\end{itemize}

\section{Download}
You will probably know where to download m23 as you are reading this manual ;). But if not:\\
The ISO to burn the m23 server installation CD, documentation and Debian packages (at a later time) can be found under: \underline{http://m23.sf.net}.



\section{Support}
\subsection{m23 community support}
You can get the m23 community support on voluntary basis for free. "Voluntary" means that you don't have the right to get your problems with m23 solved under all circumstances. The m23 community support covers problems with the installation of m23 in standard configurations and the fixing of bugs in m23 only.\\

You need to have good knowlegde of Debian and you have to be willing to give accurate fault tracing and to solve the problem together with the m23 developer, if you want to engage the m23 community support.\\

If you need more help or other services for m23, you should use m23's commercial support.\\

At http://m23.sf.net you can find:
\begin{itemize}
\item User forum
\item Documentation
\item ISO images for burning m23 server installation CDs
\item SourceCodes of all m23 components
\item etc.
\end{itemize}



\subsection{Commercial support for m23 and other OSS solutions}
The commercial support (German only) includes services for m23 and other OpenSource related topics:
\begin{itemize}
\item m23 trainings for developers and administrators in small learning groups
\item Development of individual m23 extensions and customisations
\item Consulting: Planning of clients, servers and networks to use with m23 and general OpenSource consulting
\item Creation of virtual machines and virtual appliances
\item Design and development of websites based on the content management system (CMS) devalcms
\item Services for the website capturing software khtml2png
\item Creation and planning of online voting platforms on the basis of the GRDM system
\item Trainings for Linux, OpenSource, script programming with BASH, awk, sed, grep and co. as well as schoolings for HTML, PHP and MySQL
\end{itemize}
Additional information about the services can be found at \underline{www.goos-habermann.de}.



\section{Information about the manual}
With the new m23 release, the manual takes the next step of its development. To make it easier for the developers, a big part of the manual is generated from the online help of the m23 web interface. The advantages of this method are that the documentation does not have to be created twice (interface + manual) and that this is possible in all languages the web interface supports.