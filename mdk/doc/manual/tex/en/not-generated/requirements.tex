\chapter{Requirements for m23}
To integrate the software deployment system m23 into your network, you need the following components:
\begin{itemize}
\item m23 server
\item m23 client(s)
\item Internet access
\item PC with webbrowser
\end{itemize}

\section{The m23 server}
The m23 server takes care of the complete software deployment and performs the following tasks:
\begin{itemize}
\item Generation of the web interface
\item Compilation of installation scripts "on-the-fly"
\item Software package cache
\item Database
\item Boot server
\item DHCP server
\end{itemize}
You should dimension your m23 server with enough RAM and an adequate processor. If you want to install a large variety of software packages on your clients, you need a harddisk with a lot of space to make caching of these packages possible.
\begin{itemize}
\item CPU: 1GHz minimum
\item RAM: 1GB minimum
\item Harddisk: 10GB minimum
\item Network card: recommended 100MBit/1GBit
\end{itemize}

\section{The m23 clients}
For clients, the recommendations are roughly the same. To facilitate software deployment for you, you should use wake-on-lan-enabled clients which can be booted for installation and switched off afterwards by m23. It is a good idea to use network cards containing a bootrom to allow the clients to be installed remotely. If the network card does not have this feature, a boot CD/DVD or USB flash drive are required at the beginning, for the installation of the operating system. m23 supports the PXE boot standard.

\section{Internet access}
m23 needs access to the internet to fetch the software packages that can be installed on the clients.

\section{PC with webbrowser}
For your administration, you need a PC with a web browser to access the m23 server.

\section{The server installation CD}
To install the m23 server you need the m23 server installation CD which can be downloaded for free from  \underline{http://m23.sf.net}. Burn the ISO file on a CD and boot the server from it.
