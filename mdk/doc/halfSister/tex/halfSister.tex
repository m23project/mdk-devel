\documentclass[a4paper,10pt]{article}


%opening
\title{test}
\author{Hauke Goos-Habermann}

\begin{document}

\maketitle

\section{Preamble}
halfSister is a new abstract method of m23 to support <b>quasi arbitrary Linux distributions</b> and making them installable and administrable with m23.

gfx/halfSister-Schema-eng.png

The support is reached by doing the base installation from a compressed and pre-configured archive. This way the usage of distribution specific tools like debootstrap or rpmstrap for installing a minimal system is avoided. A similar routine is used for speeding up the installation of Ubuntu with m23.

And the simplification goes on: To hide the differences between the distributions from m23, there is an adequate administration tool for the particular distributions that uses a consistent instruction set for all administrative tasks. That's why m23 doesn't need to "know" how  a specific package manager works and how it is used. It doesn't matter if the native package manager is yum, APT or something else. On every system, the command for updating the package list is "<code>m23HSAdmin pkgUpdateCache</code>" and the installation of the package "mc" is done by "<code>m23HSAdmin pkgInstall mc</code>". There are other commands of m23HSAdmin that call the native tools of the distribution for setting up the network, the installation of the bootmanager or for adjusting the system language, etc.


\section{The }

\end{document}
